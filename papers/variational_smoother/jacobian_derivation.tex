%% 
%% Copyright 2007-2024 Elsevier Ltd
%% 
%% This file is part of the 'Elsarticle Bundle'.
%% ---------------------------------------------
%% 
%% It may be distributed under the conditions of the LaTeX Project Public
%% License, either version 1.3 of this license or (at your option) any
%% later version.  The latest version of this license is in
%%    http://www.latex-project.org/lppl.txt
%% and version 1.3 or later is part of all distributions of LaTeX
%% version 1999/12/01 or later.
%% 
%% The list of all files belonging to the 'Elsarticle Bundle' is
%% given in the file `manifest.txt'.
%% 
%% Template article for Elsevier's document class `elsarticle'
%% with numbered style bibliographic references
%% SP 2008/03/01
%% $Id: elsarticle-template-num.tex 249 2024-04-06 10:51:24Z rishi $
%%
\documentclass[preprint,12pt]{elsarticle}

%% Use the option review to obtain double line spacing
%% \documentclass[authoryear,preprint,review,12pt]{elsarticle}

%% Use the options 1p,twocolumn; 3p; 3p,twocolumn; 5p; or 5p,twocolumn
%% for a journal layout:
%% \documentclass[final,1p,times]{elsarticle}
%% \documentclass[final,1p,times,twocolumn]{elsarticle}
%% \documentclass[final,3p,times]{elsarticle}
%% \documentclass[final,3p,times,twocolumn]{elsarticle}
%% \documentclass[final,5p,times]{elsarticle}
%% \documentclass[final,5p,times,twocolumn]{elsarticle}

%% For including figures, graphicx.sty has been loaded in
%% elsarticle.cls. If you prefer to use the old commands
%% please give \usepackage{epsfig}

%% The amssymb package provides various useful mathematical symbols
\usepackage{amssymb}
%% The amsmath package provides various useful equation environments.
\usepackage{amsmath}
\usepackage{bm}
%% The amsthm package provides extended theorem environments
%% \usepackage{amsthm}

%% The lineno packages adds line numbers. Start line numbering with
%% \begin{linenumbers}, end it with \end{linenumbers}. Or switch it on
%% for the whole article with \linenumbers.
\usepackage{lineno}

%% Package for strikeout
\usepackage[normalem]{ulem}

% For creating tree diagrams
\usepackage{tikz}
\usetikzlibrary{shapes.geometric, arrows, positioning, trees}

% For writing algorithms
\usepackage[linesnumbered,ruled,vlined]{algorithm2e}

% Make \paragraph have its own line
\usepackage{indentfirst}
\usepackage{titlesec}

\titleformat{\paragraph}
  {\normalfont\itshape}{\theparagraph}{1em}{}
\titlespacing*{\paragraph}
  {0pt}{3.25ex plus 1ex minus .2ex}{1.5ex plus .2ex}

% Define commands for contributors to comment
\usepackage[svgnames]{xcolor}
\newcommand{\pab}[1]{\textcolor{blue}{{\scriptsize PAB:} #1}}
\newcommand{\rhs}[1]{\textcolor{violet}{{\scriptsize RHS:} #1}}

\usepackage{empheq}
\usepackage{xpatch}
\makeatletter
\newcommand{\colorboxed}[1]{\fcolorbox{blue}{white}{\m@th$\displaystyle#1$}}
\xpatchcmd{\@Aboxed}{\boxed}{\colorboxed}{}{}
\makeatother

% other custom commands
\newcommand{\Eq}[1]{Eq.~\eqref{#1}}
\newcommand{\Fig}[1]{Fig.~\ref{#1}}
\newcommand{\be}{\begin{equation}}
\newcommand{\ee}{\end{equation}}

\renewcommand{\vec}[1]{\bm{#1}}
\DeclareMathOperator*{\argmax}{arg\,max}
\DeclareMathOperator*{\argmin}{arg\,min}

\journal{???}

\begin{document}

\begin{frontmatter}

%% Title, authors and addresses

%% use the tnoteref command within \title for footnotes;
%% use the tnotetext command for theassociated footnote;
%% use the fnref command within \author or \affiliation for footnotes;
%% use the fntext command for theassociated footnote;
%% use the corref command within \author for corresponding author footnotes;
%% use the cortext command for theassociated footnote;
%% use the ead command for the email address,
%% and the form \ead[url] for the home page:
%% \title{Title\tnoteref{label1}}
%% \tnotetext[label1]{}
%% \author{Name\corref{cor1}\fnref{label2}}
%% \ead{email address}
%% \ead[url]{home page}
%% \fntext[label2]{}
%% \cortext[cor1]{}
%% \affiliation{organization={},
%%             addressline={},
%%             city={},
%%             postcode={},
%%             state={},
%%             country={}}
%% \fntext[label3]{}

\title{Formulation of Gradient and Hessian Terms of the Distortion-Dilation Term}

%% use optional labels to link authors explicitly to addresses:
%% \author[label1,label2]{}
%% \affiliation[label1]{organization={},
%%             addressline={},
%%             city={},
%%             postcode={},
%%             state={},
%%             country={}}
%%
%% \affiliation[label2]{organization={},
%%             addressline={},
%%             city={},
%%             postcode={},
%%             state={},
%%             country={}}

\author[inl]{Patrick Behne} %% Author name

%% Author affiliation
\affiliation[inl]{organization={Idaho National Laboratory},%Department and Organization
            addressline={1955 N. Fremont Ave}, 
            city={Idaho Falls},
            postcode={83415}, 
            state={Idaho},
            country={United States}}

%% Abstract
%\begin{abstract}
%\end{abstract}

%%Graphical abstract
%\begin{graphicalabstract}
%\includegraphics{grabs}
%\end{graphicalabstract}

%%Research highlights
%\begin{highlights}
%\item Research highlight 1
%\item Research highlight 2
%\end{highlights}

%% Keywords
%\begin{keyword}
%\end{keyword}

\end{frontmatter}

%% Add \usepackage{lineno} before \begin{document} and uncomment 
%% following line to enable line numbers
\linenumbers

%% main text
%%

%%%%%%%%%%%%%%%%%%%%%%%%%%%%%%%%%%%%%%%%%%%%%%%%%%%%%%%%%%%%%%%%%%%%%%%%%%%%%%%%%%%%%%%%%%%%%%%%%%%%%%%%%%%%%%%%%%%%%%

%\nolinenumbers
\section{Introduction}
\label{sec:intro}

The distortion-dilation functional of an $n$-dimensional (i.e., 2D or 3D) mesh to minimize is:
\be
\Aboxed{I_h(\vec{R}) = \sum_{c=1}^{N\text{cells}} \int_{\hat{\Omega}_c} E_\theta (S_c(\vec{R})) d \vec{\xi}} \,,
\ee
where:
\begin{itemize}
	\item $\vec{R} \in \mathcal{R}^{n \times N_\text{nodes}}$ is a vector containing the node locations of the mesh in the physical domain
	\item $S(\vec{R})$ is the Jacobian of the mapping from the physical cell to the reference cell. The dependence of this mapping on the node locations $\vec{R}$ is made explicit.
	\item $\vec{\xi}$ denotes the multidimensional reference coordinates.
	\item $E_\theta (S(\vec{R}))$ is the distortion-dilation function.
\end{itemize}

The distortion-dilation function is given by
\be
\Aboxed{E_\theta(S(\vec{R})) = (1 - \theta) \beta(S(\vec{R})) + \theta \mu(S(\vec{R}))} \,,
\ee
where $\theta$ is a weight between zero and one, and $\beta$ and $\mu$ are measures of element distortion and dilation, given by:
\be
\Aboxed{\beta(S(\vec{R})) = \frac{\left(\frac{1}{n} \text{tr}(S^TS(\vec{R}))\right)^{\frac{n}{2}}}{\chi_\epsilon(|S(\vec{R})|)}}
\ee
and
\be
\Aboxed{\mu(S(\vec{R})) = \frac{v + \frac{|S(\vec{R})|^2}{v}}{2\chi_\epsilon(|S(\vec{R})|}} \,,
\ee
where $n$ is the dimension of the mesh, $|S|$ denotes the determinant of $S$, and $v$ is the arithmetic average value of the element-averaged $|S|$.
The function $\chi_\epsilon(|S|)$ is used in place of $|S|$ such that the metrics are well-defined for degenerate and folded elements:
\be
\Aboxed{\chi_\epsilon(|S|) = \frac{1}{2} \left( |S| + \sqrt{\epsilon^2 + |S|^2} \right)} \,,
\ee
where $\epsilon$ is a small number that prevents $\chi_\epsilon$ from being zero.
For brevity, $\chi_\epsilon(|S|$ and it's first ($\chi_\epsilon^\prime(|S|)$) and second ($\chi_\epsilon^{\prime \prime}(|S|)$) derivatives with respect to det$(S)$ will be abbreviated by $\chi_\epsilon$, $\chi_\epsilon^\prime$, and $\chi_\epsilon^{\prime \prime}$, respectively.
These functions will never take parameters other than det$(S)$.

\section{Minimization Proceedure}

We utilize Newton's method to solve the following minimization problem:
\be
\vec{R}_\text{smooth} = \argmin_{\vec{R}} I_h(\vec{R}) \,.
\ee
At the minimum of $I_h$, the gradient of $I_h$ with respect to the components of $\vec{R}$ will be zero:
\be
\nabla_{\vec{R}} I_h(\vec{R}) \rvert_{\vec{R}_\text{smooth}} = \vec{0} \,.
\ee
Newton's method can be used to find $\vec{R}_\text{smooth}$.
The update iteration $k$+1 is:
\be
\Aboxed{\vec{R}^{k+1} = \vec{R}^k - H^{-1}(I_h(\vec{R}^k)) \nabla_{\vec{R}} I_h(\vec{R}^k)} \,.
\ee
where $H(I_h)$ denotes the Jacobian of the gradient of $I_h$ (i.e., the Hessian):
\be
H(I_h) = \nabla_{\vec{R}} \nabla_{\vec{R}} I_h \,.
\ee
It is clear that in order to use Newton's method, we must first write down expressions for the gradient and Hessian of $I_h$.

\section{Matrix Calculus Identities}
\label{sec:matrix_calculus_identities}

Some matrix calculus identities are important in this work.
In this context, $S$ is a square matrix.
The derivative of a scalar-valued function $f(S)$ with respect to $S$ is a matrix with the $k,l$ entry given by
\be
\left( \frac{\partial f(S)}{\partial S} \right)_{kl} = \frac{\partial f(S)}{\partial S_{kl}} \,.
\ee
The following identities hold for the matrix derivatives of $f(S) = |S|$ and $f(S) = \text{tr}(S^TS)$:
\be
\Aboxed{\frac{\partial |S|}{\partial S} = |S| S^{-T}} \,,
\ee
\be
\Aboxed{\frac{\partial \text{tr}(S^TS)}{\partial S} = 2S} \,.
\ee

The derivative of a matrix-valued function $F(S)$ with respect to $S$ is a fourth-order tensor with the $i,j,k,\ell$ entry given by
\be
\left( \frac{\partial F(S)}{\partial S} \right)_{ijk\ell} = \frac{\partial F(S)_{ij}}{\partial S_{k\ell}} \,.
\ee
The following identity holds for the matrix derivative of $S^{-T}$ with respect to $S$:
\be
\Aboxed{\left( \frac{\partial S^{-T}}{\partial S} \right)_{ijk\ell} = - S^{-1}_{li} S^{-1}_{jk}}\,.
\ee

\section{Gradient of $I_h$}

The gradient of $I_h$ is given by
\be
\begin{align*}
\nabla_{\vec{R}} I_h(\vec{R}) &= \sum_{c=1}^{N\text{cells}} \int_{\hat{\Omega}_c}  \nabla_{\vec{R}} E_\theta (S_c(\vec{R})) d \vec{\xi} \\
&= \sum_{c=1}^{N\text{cells}} \int_{\hat{\Omega}_c}  \left( (1 - \theta) \nabla_{\vec{R}} \beta (S_c(\vec{R})) + \theta \nabla_{\vec{R}} \mu (S_c(\vec{R})) \right) d \vec{\xi} \,.
\end{align*}
\ee
It follows that we need to compute $\nabla_{\vec{R}} \beta$ and $\nabla_{\vec{R}} \mu$.

\subsection{$\nabla_{\vec{R}} \beta$}
\label{sec:grad_beta}

By definition,
\be
\nabla_{\vec{R}} \beta(\vec{R}) = \nabla_{\vec{R}} \frac{\left(\frac{1}{n} \text{tr}(S^TS(\vec{R}))\right)^{\frac{n}{2}}}{\chi_\epsilon(|S(\vec{R})|)} \,.
\ee
Denoting the gradient with respect to $\vec{R}$ (i.e., $\nabla_{\vec{R}}$) by $\partial / \partial \vec{R}$, and employing chain rule,
\be \label{eq:grad_beta}
\begin{align*}
\nabla_{\vec{R}} \beta(\vec{R}) &= \frac{\partial}{\partial \vec{R}} \beta(\vec{R}) \\
&= \frac{\partial \beta(\vec{R})}{\partial S} \frac{\partial S}{\partial \vec{R}} \,,
\end{align*}
\ee
or element-wise,
\be \label{eq:grad_beta_element}
\begin{align*}
\nabla_{\vec{R}_\ell} \beta(\vec{R}) &= \frac{\partial}{\partial \vec{R}_\ell} \beta(\vec{R}) \\
&= \sum_a \sum_b \frac{\partial \beta(\vec{R})}{\partial S_{ab}} \frac{\partial S_{ab}}{\partial \vec{R}_\ell} \,.
\end{align*}
\ee

First, we will compute the term $\partial S / \partial \vec{R}$.
In libMesh, the Jacobian matrix within cell $c$, $S(\vec{R}_c)$, has the following representation:
\be
S(\vec{R}_c) = \sum_{m=1}^{N_{\text{nodes}, c}} \vec{R}_{c,m} (\nabla_{\vec{\xi}} \phi_m(\vec{\xi}))^T \,,
\ee
where $\vec{R}_{c,m} \in \mathcal{R}^n$ are the physical coordinates of the $m$-th node in cell $c$ and $\phi_m$ is the spatial basis function for the $m$-th node.
Note that this Jacobian matrix is of shape $(n, n)$.
The expression for element $i,j$ is:
\be
S(\vec{R}_c)_{ij} = \sum_{m=1}^{N_{\text{nodes},c}} R_{c,m,i} \partial_{\xi_j} \phi_m(\vec{\xi}) \,,
\ee
where $R_{c,m,i}$ denotes the coordinate in the $i$-th dimension of node $\vec{R}_{c,m}$ and $\partial_{\xi_j}$ denotes the $j$-th component of the gradient operator $\nabla_\vec{\xi}$.
Using this definition, we see that
\be \label{eq:del_S_del_R}
\Aboxed{\frac{\partial S(\vec{R}_c)_{i,j}}{\partial R_{c,m,k}} = \delta_{ik} \partial_{\xi_j} \phi_m(\vec{\xi})} \,.
\ee
Because \Eq{eq:del_S_del_R} holds for $i,j \in [0, n)$ and $k \in [0, n \times N_\text{nodes})$, $\partial S / \partial \vec{R}$ is a third-order tensor (i.e., the derivative of a matrix with respect to a vector).

Next, we will compute the $\partial \beta / \partial S$ term.
Note that in all derivations, we use product + chain rules as opposed to quotient rule because quotient rule sucks.
Quotient rule is fake math invented by Big Mathematics so they had extra material to put in textbooks and sell them for more.
On top of that, anyone who prefers to use quotient rule instead of writing $1 / \chi_\epsilon = \chi_\epsilon^{-1}$ is obviously delusional and should not be trusted.
Also note that we utilized the identities from Section \ref{sec:matrix_calculus_identities} when using chain rule.
\be \label{eq:del_beta_del_S}
\begin{align*}
\frac{\partial \beta}{\partial S} &= \frac{\partial}{\partial S} \frac{\left(\frac{1}{n} \text{tr}(S^TS)\right)^{\frac{n}{2}}}{\chi_\epsilon} \\
&= \frac{\frac{n}{2} \left( \frac{1}{n} \text{tr}(S^TS) \right)^{\frac{n}{2}-1} \frac{1}{n} 2S}{\chi_\epsilon} - \frac{\left( \frac{1}{n} \text{tr}(S^TS) \right)^{\frac{n}{2}}}{\chi_\epsilon^2} \frac{\partial \chi_\epsilon (|S|)}{\partial S} \\
&= \Aboxed{\left( \frac{\left( \frac{1}{n} \text{tr}(S^TS) \right)^{\frac{n}{2} - 1}}{\chi_\epsilon} \right) S - \left( \frac{\left( \frac{1}{n} \text{tr}(S^TS) \right)^\frac{n}{2}}{\chi_\epsilon^2} \chi_\epsilon^\prime |S| \right) S^{-T}} \,, \\
\end{align*}
\ee
where we have used
\be \label{eq:chi_prime}
\Aboxed{\frac{\partial \chi_\epsilon}{\partial S} = \underbrace{\frac{1}{2} \left( 1 + \frac{|S|}{\sqrt{\epsilon^2 + |S|^2}} \right)}_{\chi_\epsilon^\prime} |S|S^{-T}} \,.
\ee

The results of Eqs.~\eqref{eq:del_S_del_R} and \eqref{eq:del_beta_del_S} can be substituted into \Eq{eq:grad_beta} and \Eq{eq:grad_beta_element} to obtain $\nabla_{\vec{R}} \beta(\vec{R})$ in terms of $S$ and $\phi_l$.
Note that \Eq{eq:del_S_del_R} is a rank 3 tensor of dimension $(n, n, n \times N_\text{nodes})$ and \Eq{eq:del_beta_del_S} is a matrix of dimension $(n, n)$:
\be
\nabla_{\vec{R}} \beta(\vec{R}) = \underbrace{\frac{\partial \beta(\vec{R})}{\partial S}}_{(n, n)} \underbrace{\frac{\partial S}{\partial \vec{R}}}_{(n, n, n \times N_\text{nodes})}
\ee
The expected result for the gradient of a scalar function is a vector, which can be obtained by first multiplying common dimensions together and then summing over them to obtain a vector of length $n \times N_\text{nodes}$.
Using Einstein notation, the $\ell$-th entry of this vector is given by
\be
\Aboxed{\left( \nabla_{\vec{R}} \beta \right)_\ell = \left( \frac{\partial \beta}{\partial S} \right)_{ab} \left( \frac{\partial S}{\partial \vec{R}} \right)_{ab \ell}} \,.
\ee
This result is in agreement with \Eq{eq:grad_beta_element}.

\subsection{$\nabla_{\vec{R}} \mu$}
\label{sec:grad_mu}

By definition,
\be
\nabla_{\vec{R}} \mu(\vec{R}) = \nabla_{\vec{R}} \frac{v + \frac{|S(\vec{R})|^2}{v}}{2\chi_\epsilon(|S(\vec{R})|)} \,.
\ee
Using chain rule in the same manner as the previous section,
\be \label{eq:grad_mu}
\begin{align*}
\nabla_{\vec{R}} \mu(\vec{R}) &= \frac{\partial}{\partial \vec{R}} \mu(\vec{R}) \\
&= \frac{\partial \mu(\vec{R})}{\partial S} \frac{\partial S}{\partial \vec{R}} \,.
\end{align*}
\ee

The $\partial S / \partial \vec{R}$ term is given by \Eq{eq:del_S_del_R}.
Using product rule and the Section \ref{sec:matrix_calculus_identities} identities for chain rule,
\be \label{eq:del_mu_del_S}
\begin{align*}
\frac{\partial \mu}{\partial S} &= \frac{\partial}{\partial S} \left( \frac{v + \frac{|S|^2}{v}}{2\chi_\epsilon} \right) \\
				&= \left( \frac{|S|^2 }{v\chi_\epsilon} \right) S^{-T} - \left[ \frac{v + \frac{|S|^2}{v}}{2 \chi_\epsilon^2} \right] \frac{\partial \chi_\epsilon}{\partial S}  \\
				&= \left( \frac{|S|^2 }{v\chi_\epsilon} \right) S^{-T} - \left( \left[ \frac{v + \frac{|S|^2}{v}}{2 \chi_\epsilon^2} \right] \chi_\epsilon^\prime |S| \right) S^{-T} \\
				&= \frac{|S|}{\chi_\epsilon} \left( \frac{|S| }{v} - \left[ \frac{v + \frac{|S|^2}{v}}{2 \chi_\epsilon} \right] \chi_\epsilon^\prime \right) S^{-T} \\
				&= \Aboxed{ \underbrace{\frac{1}{2v \chi_\epsilon} \left( -\chi_\epsilon^\prime |S|^3 + 2 \chi_\epsilon |S|^2 - v^2 \chi_\epsilon^\prime |S| \right)}_{:= \alpha(S)}  S^{-T} } \,,
\end{align*}
\ee
where $\chi_\epsilon^\prime$ is given by \Eq{eq:chi_prime}.
The results of Eqs.~\eqref{eq:del_S_del_R} and \eqref{eq:del_mu_del_S} can be substituted into \Eq{eq:grad_mu} to obtain $\nabla_{\vec{R}} \mu(\vec{R})$ in terms of $S$ and $\phi_l$.
Note that \Eq{eq:del_S_del_R} is a rank 3 tensor of dimension $(n, n, n \times N_\text{nodes})$ and \Eq{eq:del_beta_del_S} is a matrix of dimension $(n, n)$:
\be
\nabla_{\vec{R}} \mu(\vec{R}) = \underbrace{\frac{\partial \mu(\vec{R})}{\partial S}}_{(n,n)} \underbrace{\frac{\partial S}{\partial \vec{R}}}_{(n,n,n \times N_\text{nodes})} \,.
\ee
The expected result for the gradient of a scalar function is a vector, which can be obtained by first multiplying common dimensions together and then summing over them to obtain a vector of length $n \times N_\text{nodes}$.
Using Einstein notation, the $\ell$-th entry of this vector is given by
\be
\Aboxed{\left( \nabla_{\vec{R}} \mu \right)_\ell = \left( \frac{\partial \mu}{\partial S} \right)_{ab} \left( \frac{\partial S}{\partial \vec{R}} \right)_{ab \ell}} \,.
\ee

\section{Hessian of $I_h$}

The Hessian of $I_h$ is given by
\be
\begin{align*}
	H(I_h(\vec{R})) &= \sum_{c=1}^{N\text{cells}} \int_{\hat{\Omega}_c}  H(E_\theta (S_c(\vec{R}))) d \vec{\xi} \\
	&= \sum_{c=1}^{N\text{cells}} \int_{\hat{\Omega}_c}  \left( (1 - \theta) H(\beta (S_c(\vec{R}))) + \theta H(\mu (S_c(\vec{R}))) \right) d \vec{\xi} \,.
\end{align*}
\ee
Computing the Hessian of $I_h$ entails the computation of the Jacobian of the gradient of $I_h$, or loosely speaking,
\be
H(I_h) = \nabla_{\vec{R}} \nabla_{\vec{R}} I_h \,.
\ee
We computed the gradient of $I_h$ ($\nabla_{\vec{R}} I_h$) in the previous section.
To compute the Hessian, we simply need to take the Jacobian of those results.

\subsection{$\nabla_{\vec{R}} \nabla_{\vec{R}} \beta$}

From Section~\ref{sec:grad_beta}, we have
\be
\begin{align*}
\nabla_{\vec{R}} \beta(\vec{R}) &= \frac{\partial \beta(\vec{R})}{\partial S} \frac{\partial S}{\partial \vec{R}} \,, \\
\frac{\partial S(\vec{R}_c)_{i,j}}{\partial R_{c,m,k}} &= \delta_{ik} \partial_{\xi_j} \phi_m(\vec{\xi}) \,, \\
\frac{\partial \beta(\vec{R})}{\partial S_{k \ell}} &= \left( \frac{\left( \frac{1}{n} \text{tr}(S^TS) \right)^{\frac{n}{2} - 1}}{\chi_\epsilon} \right) S_{k \ell} - \left( \frac{\left( \frac{1}{n} \text{tr}(S^TS) \right)^\frac{n}{2}}{\chi_\epsilon^2} \chi_\epsilon^\prime |S| \right) S^{-T}_{k \ell} \,.
\end{align*}
\ee
Applying another derivative with respect to $\vec{R}$, the Hessian of $\beta$ is expressed as:
\be
\label{eq:hessian_beta}
\begin{align*}
	H(\beta(\vec{R})) &= \frac{\partial}{\partial \vec{R}} \left( \frac{\partial \beta(\vec{R})}{\partial S} \frac{\partial S}{\partial \vec{R}} \right) \\
	&= \frac{\partial}{\partial \vec{R}} \left( \frac{\partial \beta(\vec{R})}{\partial S} \right) \frac{\partial S}{\partial \vec{R}} + \frac{\partial \beta(\vec{R})}{\partial S} \frac{\partial}{\partial \vec{R}} \left( \frac{\partial S}{\partial \vec{R}} \right) \\
	&= \frac{\partial}{\partial S} \left( \frac{\partial \beta(\vec{R})}{\partial S} \right) \frac{\partial S}{\partial \vec{R}} \frac{\partial S}{\partial \vec{R}} + \frac{\partial \beta(\vec{R})}{\partial S} \underbrace{\frac{\partial^2 S}{\partial \vec{R}^2}}_{= 0 \text{ by } \Eq{eq:del_S_del_R}} \\
	&= \frac{\partial^2 \beta(\vec{R})}{\partial S^2} \frac{\partial S}{\partial \vec{R}} \frac{\partial S}{\partial \vec{R}} \,.
\end{align*}
\ee
Entry-wise, this is
\be
\label{eq:hessian_beta_element}
\begin{align*}
	H(\beta(\vec{R}))_{\ell p}  &= \left( \frac{\partial}{\partial \vec{R}} \left( \frac{\partial}{\partial \vec{R}} \beta(\vec{R}) \right)_\ell \right)_p \\
	&= \frac{\partial}{\partial R_p} \frac{\partial}{\partial R_\ell} \beta(\vec{R}) \\
	&= \frac{\partial}{\partial R_p} \sum_a \sum_b \left( \frac{\partial \beta(\vec{R})}{\partial S_{ab}} \frac{\partial S_{ab}}{\partial R_\ell} \right) \\
	&= \sum_a \sum_b \left[ \frac{\partial}{\partial R_p} \left( \frac{\partial \beta(\vec{R})}{\partial S_{ab}} \right) \frac{\partial S_{ab}}{\partial R_\ell} + \frac{\partial \beta(\vec{R})}{\partial S_{ab}} \frac{\partial}{\partial R_p} \left( \frac{\partial S_{ab}}{\partial R_\ell} \right) \right] \\
	&= \sum_a \sum_b \left[ \sum_i \sum_j \frac{\partial}{\partial S_{ij}} \left( \frac{\partial \beta(\vec{R})}{\partial S_{ab}} \right) \frac{\partial S_{ab}}{\partial R_\ell} \frac{\partial S_{ij}}{\partial R_p} + \frac{\partial \beta(\vec{R})}{\partial S_{ab}} \underbrace{\frac{\partial^2 S_{ab}}{\partial R_p \partial R_\ell}}_{= 0 \text{ by } \Eq{eq:del_S_del_R}} \right] \\
	&= \sum_a \sum_b \frac{\partial S_{ab}}{\partial R_\ell} \sum_i \sum_j \left( \frac{\partial^2 \beta(\vec{R})}{\partial S_{ij} \partial S_{ab}} \right) \left( \frac{\partial S_{ij}}{\partial R_p} \right) \,.
\end{align*}
\ee
The term $\partial S / \partial \vec{R}$ is given by \Eq{eq:del_S_del_R}.
The term $\partial^2 \beta / \partial S^2$ is given by differentiating \Eq{eq:del_beta_del_S} with respect to $S$.
Using the identities from Section \ref{sec:matrix_calculus_identities},
\be
\begin{align*}
	\frac{\partial^2 \beta(\vec{R})}{\partial S_{ab} \partial S_{ij}}
&= \frac{\partial}{\partial S_{ab}} \left[ \left( \frac{\left( \frac{1}{n} \text{tr}(S^TS) \right)^{\frac{n}{2} - 1}}{\chi_\epsilon} \right) S_{ij} - \left( \frac{\left( \frac{1}{n} \text{tr}(S^TS) \right)^\frac{n}{2}}{\chi_\epsilon^2} \chi_\epsilon^\prime |S| \right) S^{-T}_{ij} \right] \\
&= \frac{1}{\chi_\epsilon} \left[ \frac{n-2}{2} \left(\frac{1}{n} \text{tr}(S^TS)\right)^{\frac{n}{2}-2} \frac{2S_{ab}}{n} S_{ij} + \left(\frac{1}{n} \text{tr}(S^TS)\right)^{\frac{n}{2}-1} I_{ia} I_{jb} \right] \\
&- \frac{1}{\chi_\epsilon^2} \left(\frac{1}{n} \text{tr}(S^TS)\right)^{\frac{n}{2}-1} \chi_\epsilon^\prime |S| S^{-T}_{ab} S_{ij} \\
&- \frac{1}{\chi_\epsilon^2} \left[ \frac{n}{2} \left(\frac{1}{n} \text{tr}(S^TS)\right)^{\frac{n}{2}-1} \frac{2S_{ab}}{n}  \chi_\epsilon^\prime |S| S^{-T}_{ij} \right. \\
&+ \left. \left(\frac{1}{n} \text{tr}(S^TS)\right)^{\frac{n}{2}} \chi_\epsilon^{\prime \prime} |S|^2 S^{-T}_{ab} S^{-T}_{ij} \right. \\
&+ \left. \left(\frac{1}{n} \text{tr}(S^TS)\right)^{\frac{n}{2}} \chi_\epsilon^\prime |S| S^{-T}_{ab} S^{-T}_{ij} \right] \\
&+ \frac{2}{\chi_\epsilon^3} \left[ \left(\frac{1}{n} \text{tr}(S^TS)\right)^{\frac{n}{2}} \chi_\epsilon^\prime |S| S^{-T}_{ab} \chi_\epsilon^\prime |S| S^{-T}_{ij} \right] \\
&+ \left( \frac{\left( \frac{1}{n} \text{tr}(S^TS) \right)^\frac{n}{2}}{\chi_\epsilon^2} \chi_\epsilon^\prime |S| \right) S^{-1}_{bi} S^{-1}_{ja} \\
	\Aboxed{&= \left[ \left(\frac{1}{n} \text{tr}(S^TS)\right)^{\frac{n}{2}-1} \frac{1}{\chi_\epsilon} \right] I_{ia} I_{jb} + \left[ \frac{n-2}{n \chi_\epsilon} \left(\frac{1}{n} \text{tr}(S^TS)\right)^{\frac{n}{2}-2} \right] S_{ab} S_{ij}} \\
\Aboxed{&- \left[ \left(\frac{1}{n} \text{tr}(S^TS)\right)^{\frac{n}{2}-1}  \frac{\chi_\epsilon^\prime |S|}{\chi_\epsilon^2} \right] S^{-1}_{ba} S_{ij}
- \left[ \left(\frac{1}{n} \text{tr}(S^TS)\right)^{\frac{n}{2}-1} \frac{\chi_\epsilon^\prime |S|}{\chi_\epsilon^2} \right] S_{ab} S^{-1}_{ji}} \\
\Aboxed{&+ \left(\frac{1}{n} \text{tr}(S^TS)\right)^{\frac{n}{2}} \left( \frac{|S|}{\chi_\epsilon^2} \right) \left[ \frac{2 \chi_\epsilon^\prime^2 |S|}{\chi_\epsilon} 
- \chi_\epsilon^{\prime \prime} |S|
- \chi_\epsilon^\prime \right] S^{-1}_{ba} S^{-1}_{ji}} \\
\Aboxed{&- \left( \frac{\left( \frac{1}{n} \text{tr}(S^TS) \right)^\frac{n}{2}}{\chi_\epsilon^2} \chi_\epsilon^\prime |S| \right) S^{-1}_{bi} S^{-1}_{ja}} \,.
\end{align*}
\ee
where
\be
\Aboxed{\chi_\epsilon^{\prime \prime} = \frac{1}{2} \left( \frac{1}{\sqrt{\epsilon^2 + |S|^2}} - \frac{|S|^2}{(\epsilon^2 + |S|^2)^{3/2}} \right)} \,.
\ee

\subsection{$\nabla_{\vec{R}} \nabla_{\vec{R}} \mu$}

From Section~\ref{sec:grad_mu}, we have
\be
\begin{align*}
\nabla_{\vec{R}} \mu(\vec{R}) &= \frac{\partial \mu(\vec{R})}{\partial S} \frac{\partial S}{\partial \vec{R}} \,, \\
\frac{\partial S(\vec{R}_c)_{i,j}}{\partial R_{c,m,k}} &= \delta_{ik} \partial_{\xi_j} \phi_m(\vec{\xi}) \,, \\
\frac{\partial \mu(\vec{R})}{\partial S} &= \alpha(S) S^{-T} \,, \\
\alpha(S) &= \frac{1}{2v \chi_\epsilon} \left( -\chi_\epsilon^\prime |S|^3 + 2 \chi_\epsilon |S|^2 - v^2 \chi_\epsilon^\prime |S| \right) \,.
\end{align*}
\ee
Applying another derivative with respect to $\vec{R}$, the Hessian of $\mu$ is expressed as:
\be
\label{eq:hessian_mu}
\begin{align*}
	H(\mu(\vec{R})) &= \frac{\partial}{\partial \vec{R}} \left( \frac{\partial \mu(\vec{R})}{\partial S} \frac{\partial S}{\partial \vec{R}} \right) \\
	&= \frac{\partial}{\partial \vec{R}} \left( \frac{\partial \mu(\vec{R})}{\partial S} \right) \frac{\partial S}{\partial \vec{R}} + \frac{\partial \mu(\vec{R})}{\partial S} \frac{\partial}{\partial \vec{R}} \left( \frac{\partial S}{\partial \vec{R}} \right) \\
	&= \frac{\partial}{\partial S} \left( \frac{\partial \mu(\vec{R})}{\partial S} \right) \frac{\partial S}{\partial \vec{R}} \frac{\partial S}{\partial \vec{R}} + \frac{\partial \mu(\vec{R})}{\partial S} \underbrace{\frac{\partial^2 S}{\partial \vec{R}^2}}_{= 0 \text{ by } \Eq{eq:del_S_del_R}} \\
	&= \frac{\partial^2 \mu(\vec{R})}{\partial S^2} \frac{\partial S}{\partial \vec{R}} \frac{\partial S}{\partial \vec{R}} \,.
\end{align*}
\ee
Entry-wise, this is
\be
\label{eq:hessian_mu_element}
\begin{align*}
	H(\mu(\vec{R}))_{\ell p}  &= \left( \frac{\partial}{\partial \vec{R}} \left( \frac{\partial}{\partial \vec{R}} \mu(\vec{R}) \right)_\ell \right)_p \\
	&= \frac{\partial}{\partial R_p} \frac{\partial}{\partial R_\ell} \mu(\vec{R}) \\
	&= \frac{\partial}{\partial R_p} \sum_a \sum_b \left( \frac{\partial \mu(\vec{R})}{\partial S_{ab}} \frac{\partial S_{ab}}{\partial R_\ell} \right) \\
	&= \sum_a \sum_b \left[ \frac{\partial}{\partial R_p} \left( \frac{\partial \mu(\vec{R})}{\partial S_{ab}} \right) \frac{\partial S_{ab}}{\partial R_\ell} + \frac{\partial \mu(\vec{R})}{\partial S_{ab}} \frac{\partial}{\partial R_p} \left( \frac{\partial S_{ab}}{\partial R_\ell} \right) \right] \\
	&= \sum_a \sum_b \left[ \sum_i \sum_j \frac{\partial}{\partial S_{ij}} \left( \frac{\partial \mu(\vec{R})}{\partial S_{ab}} \right) \frac{\partial S_{ab}}{\partial R_\ell} \frac{\partial S_{ij}}{\partial R_p} + \frac{\partial \mu(\vec{R})}{\partial S_{ab}} \underbrace{\frac{\partial^2 S_{ab}}{\partial R_p \partial R_\ell}}_{= 0 \text{ by } \Eq{eq:del_S_del_R}} \right] \\
	&= \sum_a \sum_b \frac{\partial S_{ab}}{\partial R_\ell} \sum_i \sum_j \left( \frac{\partial^2 \mu(\vec{R})}{\partial S_{ij} \partial S_{ab}} \right) \left( \frac{\partial S_{ij}}{\partial R_p} \right) \,.
\end{align*}
\ee
The term $\partial S / \partial \vec{R}$ is given by \Eq{eq:del_S_del_R}.
The term $\partial^2 \mu / \partial S^2$ is given by differentiating \Eq{eq:del_mu_del_S} with respect to $S$.
Using the identities from Section \ref{sec:matrix_calculus_identities},
\be
\begin{align*}
\frac{\partial^2 \mu(\vec{R})}{\partial S_{ab} \partial S_{ij}}
&= \frac{\partial}{\partial S_{ab}} \alpha(S) S^{-T}_{ij} \right) \\
&= \left( \frac{\partial}{\partial S_{ab}} \alpha(S) \right) S^{-T}_{ij} + \alpha(S) \frac{\partial}{\partial S_{ab}} S^{-T}_{ij} \\
&= \frac{\partial}{\partial S_{ab}} \left( \frac{1}{2v \chi_\epsilon} \left( -\chi_\epsilon^\prime |S|^3 + 2 \chi_\epsilon |S|^2 - v^2 \chi_\epsilon^\prime |S| \right) \right) S^{-T}_{ij} + \alpha(S) S^{-1}_{bi} S^{-1}_{ja} \\
&= \left( -\frac{2 \alpha(S) \chi_\epsilon^\prime |S| S^{-T}_{ab}}{\chi_\epsilon} + \frac{1}{2v \chi_\epsilon^2} \big( -\chi_\epsilon^{\prime\prime} |S|^4 S^{-T}_{ab} - 3 \chi_\epsilon^\prime |S|^3 S^{-T}_{ab} + 2 \chi_\epsilon^\prime |S|^3 S^{-T}_{ab} \right. \\
&+ \left. 4 \chi_\epsilon |S|^2 S^{-T}_{ab} - v^2 \chi_\epsilon^\prime |S| S^{-T}_{ab} - v^2 |S|^2 \chi_\epsilon^{\prime\prime} S^{-T}_{ab} \vphantom{\frac{1}{2v \chi_\epsilon}} \big) \right) S^{-T}_{ij} + \alpha(S) S^{-1}_{bi} S^{-1}_{ja} \\
\Aboxed{&= \frac{|S|}{2v \chi_\epsilon^2} \Big( -4v \alpha(S) \chi_\epsilon \chi_\epsilon^\prime - \chi_\epsilon^{\prime\prime} |S|^3 - \chi_\epsilon^\prime |S|^2 + 4 \chi_\epsilon |S| - v^2 \left( \chi_\epsilon^\prime + |S| \chi_\epsilon^{\prime\prime} \right) \Big) S^{-1}_{ba} S^{-1}_{ji}} \\
\Aboxed{&+ \alpha(S) S^{-1}_{bi} S^{-1}_{ja}} \,.
\end{align*}
\ee

%%%%%%%%%%%%%%%%%%%%%%%%%%%%%%%%%%%%%%%%%%%%%%%%%%%%%%%%%%%%%%%%%%%%%%%%%%%%%%%%%%%%%%%%%%%%%%%%%%%%%%%%%%%%%%%%%%%%%%

%%% Use \section commands to start a section
%\section{Example Section}
%\label{sec1}
%%% Labels are used to cross-reference an item using \ref command.
%
%Section text. See Subsection \ref{subsec1}.
%
%%% Use \subsection commands to start a subsection.
%\subsection{Example Subsection}
%\label{subsec1}
%
%Subsection text.

%% Use \subsubsection, \paragraph, \subparagraph commands to 
%% start 3rd, 4th and 5th level sections.
%% Refer following link for more details.
%% https://en.wikibooks.org/wiki/LaTeX/Document_Structure#Sectioning_commands


%% Refer following link for more details.
%% https://en.wikibooks.org/wiki/LaTeX/Mathematics
%% https://en.wikibooks.org/wiki/LaTeX/Advanced_Mathematics

%%% Use a table environment to create tables.
%%% Refer following link for more details.
%%% https://en.wikibooks.org/wiki/LaTeX/Tables
%\begin{table}[t]%% placement specifier
%%% Use tabular environment to tag the tabular data.
%%% https://en.wikibooks.org/wiki/LaTeX/Tables#The_tabular_environment
%\centering%% For centre alignment of tabular.
%\begin{tabular}{l c r}%% Table column specifiers
%%% Tabular cells are separated by &
%  1 & 2 & 3 \\ %% A tabular row ends with \\
%  4 & 5 & 6 \\
%  7 & 8 & 9 \\
%\end{tabular}
%%% Use \caption command for table caption and label.
%\caption{Table Caption}\label{fig1}
%\end{table}


%%% Use figure environment to create figures
%%% Refer following link for more details.
%%% https://en.wikibooks.org/wiki/LaTeX/Floats,_Figures_and_Captions
%\begin{figure}[t]%% placement specifier
%%% Use \includegraphics command to insert graphic files. Place graphics files in 
%%% working directory.
%\centering%% For centre alignment of image.
%\includegraphics{example-image-a}
%%% Use \caption command for figure caption and label.
%\caption{Figure Caption}\label{fig1}
%%% https://en.wikibooks.org/wiki/LaTeX/Importing_Graphics#Importing_external_graphics
%\end{figure}


%%% The Appendices part is started with the command \appendix;
%%% appendix sections are then done as normal sections
%\appendix

%%% For citations use: 
%%%       \cite{<label>} ==> [1]
%
%%%
%Example citation, See \cite{lamport94}.

%% If you have bib database file and want bibtex to generate the
%% bibitems, please use
%%
%\bibliographystyle{elsarticle-num} 
%\bibliography{references}


\end{document}

\endinput
%%
%% End of file `elsarticle-template-num.tex'.
